\section{Рабочий проект}
\subsection{Спецификация компонентов и классов программы}

\subsection{Модуль shift.py}

Модуль shift.py содержит функцию для случайного сдвига изображения по осям X и Y. Модуль не содержит классов. Метод модуля - shift image. Он выполняет случайный сдвиг изображения на расстояние, не превышающее заданный процент от его размеров, с использованием аффинного преобразования. Входные данные:

\begin{itemize}
	\item image (тип PIL.Image.Image) – исходное изображение для сдвига;
	\item max shift (тип float, значение по умолчанию = 0.1) – максимальный процент сдвига от размера изображения.
\end{itemize}

Возвращаемые данные: PIL.Image.Image – изображение после сдвига.

\subsection{Модуль brightness.py}

Модуль brightness.py предоставляет функцию для изменения яркости изображения. Модуль не содержит классов. Метод модуля - change brightness. Он изменяет яркость изображения с использованием класса ImageEnhance.Brightness из библиотеки Pillow. Входные данные:

\begin{itemize}
	\item image (тип PIL.Image.Image) – исходное изображение;
	\item factor (тип float, значение по умолчанию = 1.0) – коэффициент яркости (меньше 1 – затемнение, больше 1 – осветление).
\end{itemize}

Возвращаемые данные: PIL.Image.Image – изображение с измененной яркостью.

\subsection{Модуль rotate.py}

Модуль rotate.py содержит функцию для поворота изображения на заданный угол. Модуль не содержит классов. Метод модуля - rotate image. Он выполняет поворот изображения на заданный угол с возможностью изменения размера после поворота. Входные данные:

\begin{itemize}
	\item image (тип PIL.Image.Image) – исходное изображение;
	\item angle (тип float) – угол поворота в градусах;
	\item target size (тип tuple, необязательный) – целевой размер изображения после поворота (ширина, высота).
\end{itemize}

Возвращаемые данные: PIL.Image.Image – повернутое изображение (и измененного размера, если указан target size).

\subsection{Модуль flip.py}

Модуль flip.py предоставляет функцию для отражения изображения по горизонтали или вертикали. Модуль не содержит классов. Метод модуля - flip image. Он отражает изображение по горизонтали или вертикали с использованием метода transpose из библиотеки Pillow. При некорректном значении параметра mode вызывает исключение ValueError. Входные данные:

\begin{itemize}
	\item image (тип PIL.Image.Image) – исходное изображение;
	\item mode (тип str, значение по умолчанию = 'horizontal') – режим отражения (horizontal или vertical).
\end{itemize}

Возвращаемые данные: PIL.Image.Image – отраженное изображение.

\subsection{Модуль noise.py}

Модуль noise.py содержит функцию для добавления шума к изображению в градациях серого. Модуль не содержит классов. Метод модуля - add noise. Он добавляет к изображению шум типа "гауссовский" или "соль и перец". Для обработки используется библиотека NumPy. Требует, чтобы изображение было в режиме L, иначе вызывает исключение ValueError. Входные данные:

\begin{itemize}
	\item image (тип PIL.Image.Image) – исходное изображение в режиме L (градации серого);
	\item noise type (тип str, значение по умолчанию = 'gaussian') – тип шума (gaussian или salt pepper);
	\item **kwargs – дополнительные параметры:
	\begin{itemize}
		\item для gaussian: stddev (тип float, значение по умолчанию = 10) – стандартное отклонение шума;
		\item для salt pepper: amount (тип float, значение по умолчанию = 0.05) – доля пикселей, затронутых шумом; salt vs pepper (тип float, значение по умолчанию = 0.5) – соотношение "соли" и "перца".
	\end{itemize}
\end{itemize}

Возвращаемые данные: PIL.Image.Image – изображение с добавленным шумом.

\subsection{Модуль augmentation config.py}

Модуль augmentation config.py определяет конфигурацию аугментаций. Модуль не содержит классов или методов, представляя собой словарь augmentation config. Описание конфигурации:

\begin{itemize}
	\item available augmentations (тип list) – список доступных аугментаций: noise, rotate, flip, shift, brightness;
	\item noise, rotate, flip, shift, brightness (тип dict) – настройки для каждой аугментации с полями enabled и params (например, angle range, std range);
	\item volume presets (тип dict) – предустановленные объемы генерации: 1х25, 1х50, 1х100.
\end{itemize}

Он обеспечивает централизованное управление параметрами аугментаций и их состоянием.

\subsection{Модуль pipeline.py}

Модуль pipeline.py реализует логику применения аугментаций. Он обеспечивает централизованное управление параметрами аугментаций и их состоянием. Методы модуля представлены в таблице ~\ref{table:pipeline}.

\renewcommand{\arraystretch}{0.8} % уменьшение расстояний до сетки таблицы
\begin{xltabular}{\textwidth}{|X|X|X|X|X|}
	\caption{Методы модуля pipeline.py\label{table:pipeline}}\\
	\hline 
	\centrow \setlength{\baselineskip}{0.7\baselineskip} Название метода & 
	\centrow Параметры метода &
	\centrow Возвращаемое значение & 
	\centrow Назначение метода \\ 
	\hline 
	\endfirsthead
	
	\caption*{Продолжение таблицы \ref{table:pipeline}}\\
	\hline 
	\centrow Название метода & 
	\centrow Параметры метода &
	\centrow Возвращаемое значение & 
	\centrow Назначение метода \\ 
	\hline 
	\endhead
	
	\_\_init\_\_ & Не имеет & Не имеет  & Инициализирует главное окно программы, задает его параметры, заголовок, панель меню с действиями для переключения режимов. \\ \hline 
	
	apply\_ augmentation & image (тип PIL.Image. Image) – изображение; augmentation\_ type (тип str) – тип аугментации & PIL.Image. Image – аугментированное изображение & Применяет одну аугментацию в соответствии с конфигурацией.\\
	\hline
	
	process\_ images & image (тип PIL.Image. Image) – изображение; target\_size (тип tuple) – размер; volume\_level (тип str, по умолчанию 'low') – уровень генерации & list – список аугментированных изображений & Генерирует заданное количество аугментированных версий с случайными комбинациями аугментаций.\\
	\hline
	
	process\_ images & image (тип PIL.Image. Image) – изображение; target\_size (тип tuple) – размер; volume\_level (тип str, по умолчанию 'low') – уровень генерации & list – список аугментированных изображений & Генерирует заданное количество аугментированных версий с случайными комбинациями аугментаций.\\
	\hline
	
\end{xltabular}
\renewcommand{\arraystretch}{1.0} % восстановление сетки
\vspace{-\baselineskip}


\subsection{Модуль main\_window.py}

Модуль main\_window предоставляет графический интерфейс для взаимодействия с пользователем, включая загрузку изображений, выбор директории, настройку аугментаций, предпросмотр и сохранение результатов. Константы и методы: отсутствуют.

Класс MainWindow (модуль main\_window.py).

Базовый класс: AugmentationWindow (из модуля ui.window).

Внутренние поля представлены в таблице ~\ref{table:main_window}.

\begin{xltabular}{\textwidth}{|X|X|X|}
	\caption{Внутренние поля класса MainWindow \label{table:main_window}} \\
	\hline 
	\centrow Внутреннее поле & 
	\centrow Тип & 
	\centrow Описание \\ 
	\hline 
	\endfirsthead
	
	\caption*{Продолжение таблицы \ref{table:main_window}} \\
	\hline 
	\centrow Внутреннее поле & 
	\centrow Тип & 
	\centrow Описание \\ 
	\hline 
	\endhead
	
	output\_dir & str или None & Путь к директории для сохранения результатов. \\ \hline
	image\_paths & list & Список путей к загруженным изображениям. \\ \hline
	progress\_bar & QProgressBar & Виджет для отображения прогресса обработки. \\ \hline
	scroll\_area & QScrollArea & Область прокрутки для отображения миниатюр. \\ \hline
	preview\_container & QWidget & Контейнер для размещения миниатюр аугментированных изображений. \\ \hline
	preview\_layout & QHBoxLayout & Макет для размещения миниатюр. \\ \hline
	select\_augs\_button & QPushButton & Кнопка для открытия диалога выбора аугментаций. \\ \hline
\end{xltabular}

Методы класса представлены в таблице ~\ref{table:main_window_method}

\renewcommand{\arraystretch}{0.8} % уменьшение расстояний до сетки таблицы
\begin{xltabular}{\textwidth}{|X|X|X|X|X|}
	\caption{Методы модуля pipeline.py\label{table:main_window_method}}\\
	\hline 
	\centrow \setlength{\baselineskip}{0.7\baselineskip} Название метода & 
	\centrow Параметры метода &
	\centrow Возвращаемое значение & 
	\centrow Назначение метода \\ 
	\hline 
	\endfirsthead
	
	\caption*{Продолжение таблицы \ref{table:main_window_method}}\\
	\hline 
	\centrow Название метода & 
	\centrow Параметры метода &
	\centrow Возвращаемое значение & 
	\centrow Назначение метода \\ 
	\hline 
	\endhead
	
	load\_images & Не имеет & Не имеет  & Загружает изображения из выбранной папки и отображает предпросмотр первого. \\ \hline
	
	select\_output \_folder & Не имеет & Не имеет  & Загружает изображения из выбранной папки и отображает предпросмотр первого. \\ \hline

	open\_output\_ directory & Не имеет & Не имеет  & Открывает папку с результатами в файловом менеджере. \\ \hline 

	show\_image \_preview & path (тип str) – путь к изображению & Не имеет  & Отображает предпросмотр исходного изображения в интерфейсе. \\ \hline 

	show\_ augmented\_ preview & pil\_image (тип PIL.Image.Image) – аугментированное изображение & Не имеет & Отображает предпросмотр аугментированного изображения в интерфейсе. \\ \hline
	
	apply\_ selected\_ augmentation & Не имеет & Не имеет & Выполняет аугментацию для всех загруженных изображений и сохраняет результаты. \\ \hline
	
	open\_ augmentation \_selector & Не имеет & Не имеет & Открывает диалог для настройки активных аугментаций. \\ \hline
	
	apply\_ augmentation & image (тип PIL.Image. Image) – изображение; augmentation\_ type (тип str) – тип аугментации & PIL.Image. Image – аугментированное изображение & Применяет одну аугментацию в соответствии с конфигурацией.\\
	\hline
	
	process\_ images & image (тип PIL.Image. Image) – изображение; target\_size (тип tuple) – размер; volume\_level (тип str, по умолчанию 'low') – уровень генерации & list – список аугментированных изображений & Генерирует заданное количество аугментированных версий с случайными комбинациями аугментаций.\\
	\hline
	
	process\_ images & image (тип PIL.Image. Image) – изображение; target\_size (тип tuple) – размер; volume\_level (тип str, по умолчанию 'low') – уровень генерации & list – список аугментированных изображений & Генерирует заданное количество аугментированных версий с случайными комбинациями аугментаций.\\
	\hline
	
\end{xltabular}
\renewcommand{\arraystretch}{1.0} % восстановление сетки
\vspace{-\baselineskip}

\subsection{Модуль augmentation\_selector.py}

Модуль augmentation\_selector.py предоставляет диалоговое окно для выбора активных аугментаций. Константы: отсутствуют. Метод get\_selected\_augmentations возвращает список аугментаций, отмеченных пользователем.

Внутренние поля представлены в таблице ~\ref{table:augmentation_selector}

\begin{xltabular}{\textwidth}{|X|X|X|}
	\caption{Внутренние поля класса MainWindow \label{table:augmentation_selector}} \\
	\hline 
	\centrow Внутреннее поле & 
	\centrow Тип & 
	\centrow Описание \\ 
	\hline 
	\endfirsthead
	
	\caption*{Продолжение таблицы \ref{table:main_window}} \\
	\hline 
	\centrow Внутреннее поле & 
	\centrow Тип & 
	\centrow Описание \\ 
	\hline 
	\endhead
	
	selected & set & Множество выбранных пользователем аугментаций. \\ \hline
	checkboxes & dict & Словарь, где ключ – название аугментации, значение – QCheckBox. \\ \hline
\end{xltabular}

\subsection{Модульное тестирование разработанной программной системы}

Модульное тестирование проведено для проверки корректности работы отдельных компонентов программной системы аугментации изображений. Тестирование осуществлялось с использованием библиотеки unittest языка Python, что позволило автоматизировать проверки функциональности функций и классов. Для каждого модуля сформированы таблицы тестовых наборов, содержащие описание теста, входные данные и эталонные выходные данные. Тесты выполнялись на тестовом изображении размером $100 \times 100$ пикселей, созданном программно с использованием библиотеки Pillow. Все тесты успешно пройдены, что подтверждает отсутствие ошибок в реализованных компонентах.

\subsubsection{Тестовые наборы для модуля shift.py}
\begin{xltabular}{\textwidth}{|X|X|X|}
	\caption{Тестовые наборы для функции \texttt{shift\_image} (shift.py) \label{tab:test_shift}} \\
	\hline
	\centrow Описание теста &
	\centrow Входные данные &
	\centrow Эталонные выходные данные \\
	\hline
	\endfirsthead
	
	\caption*{Продолжение таблицы \ref{tab:test_shift}} \\
	\hline
	\centrow Описание теста &
	\centrow Входные данные &
	\centrow Эталонные выходные данные \\
	\hline
	\endhead
	
	Проверка корректного сдвига изображения на 10 \% по горизонтали. & Изображение $100 \times 100$ пикселей, max\_shift=0.1, фиксированный сдвиг 10 пикселей. & Изображение $100 \times 100$ пикселей с сдвигом на 10 пикселей, заполнение пустых областей белым цветом. \\ \hline
	Проверка отсутствия изменения размеров при сдвиге. & Изображение $100 \times 100$ пикселей, max\_shift=0.05, случайный сдвиг. & Изображение $100 \times 100$ пикселей, без изменения размеров. \\ \hline
\end{xltabular}

\subsubsection{Тестовые наборы для модуля brightness.py}
\begin{xltabular}{\textwidth}{|X|X|X|}
	\caption{Тестовые наборы для функции \texttt{change\_brightness} (brightness.py) \label{tab:test_brightness}} \\
	\hline
	\centrow Описание теста &
	\centrow Входные данные &
	\centrow Эталонные выходные данные \\
	\hline
	\endfirsthead
	
	\caption*{Продолжение таблицы \ref{tab:test_brightness}} \\
	\hline
	\centrow Описание теста &
	\centrow Входные данные &
	\centrow Эталонные выходные данные \\
	\hline
	\endhead
	
	Проверка увеличения яркости на 50 \%. & Изображение $100 \times 100$ пикселей с серым цветом (128, 128, 128), factor=1.5. & Изображение $100 \times 100$ пикселей с увеличенной яркостью (192, 192, 192). \\ \hline
	Проверка сохранения исходной яркости при factor=1.0. & Изображение $100 \times 100$ пикселей с цветом (100, 100, 100), factor=1.0. & Изображение $100 \times 100$ пикселей с сохраненными значениями (100, 100, 100). \\ \hline
\end{xltabular}

\subsubsection{Тестовые наборы для модуля rotate.py}
\begin{xltabular}{\textwidth}{|X|X|X|}
	\caption{Тестовые наборы для функции \texttt{rotate\_image} (rotate.py) \label{tab:test_rotate}} \\
	\hline
	\centrow Описание теста &
	\centrow Входные данные &
	\centrow Эталонные выходные данные \\
	\hline
	\endfirsthead
	
	\caption*{Продолжение таблицы \ref{tab:test_rotate}} \\
	\hline
	\centrow Описание теста &
	\centrow Входные данные &
	\centrow Эталонные выходные данные \\
	\hline
	\endhead
	
	Проверка поворота на 90 градусов с сохранением размера. & Изображение $100 \times 100$ пикселей, angle=90, target\_size=(100, 100). & Изображение $100 \times 100$ пикселей, повернутое на 90 градусов. \\ \hline
	Проверка корректного заполнения при повороте на 45 градусов & Изображение $100 \times 100$ пикселей, angle=45, target\_size=(100, 100). & Изображение $100 \times 100$ пикселей с поворотом на 45 градусов и заполнением белым цветом. \\ \hline
\end{xltabular}

\subsubsection{Тестовые наборы для модуля flip.py}
\begin{xltabular}{\textwidth}{|X|X|X|}
	\caption{Тестовые наборы для функции \texttt{flip\_image} (flip.py) \label{tab:test_flip}} \\
	\hline
	\centrow Описание теста &
	\centrow Входные данные &
	\centrow Эталонные выходные данные \\
	\hline
	\endfirsthead
	
	\caption*{Продолжение таблицы \ref{tab:test_flip}} \\
	\hline
	\centrow Описание теста &
	\centrow Входные данные &
	\centrow Эталонные выходные данные \\
	\hline
	\endhead
	
	Проверка горизонтального отражения. & Изображение $100 \times 100$ пикселей с градиентом слева направо. & Изображение $100 \times 100$ пикселей с зеркальным отражением градиента. \\ \hline
	Проверка вертикального отражения. & Изображение $100 \times 100$ пикселей с градиентом сверху вниз. & Изображение $100 \times 100$ пикселей с вертикальным отражением градиента. \\ \hline
\end{xltabular}

\subsubsection{Тестовые наборы для модуля noise.py}
\begin{xltabular}{\textwidth}{|X|X|X|}
	\caption{Тестовые наборы для функции \texttt{add\_noise} (noise.py) \label{tab:test_noise}} \\
	\hline
	\centrow Описание теста &
	\centrow Входные данные &
	\centrow Эталонные выходные данные \\
	\hline
	\endfirsthead
	
	\caption*{Продолжение таблицы \ref{tab:test_noise}} \\
	\hline
	\centrow Описание теста &
	\centrow Входные данные &
	\centrow Эталонные выходные данные \\
	\hline
	\endhead
	
	Проверка добавления гауссовского шума. & Изображение $100 \times 100$ пикселей в градациях серого, stddev=10. & Изображение $100 \times 100$ пикселей с добавленным гауссовским шумом, дисперсия увеличена. \\ \hline
	Проверка шума "соль и перец" с долей 5 \%. & Изображение $100 \times 100$ пикселей, noise\_type='salt\_pepper', amount=0.05. & Изображение $100 \times 100$ пикселей с 5 \% пикселей заменено на черный или белый цвет. \\ \hline
\end{xltabular}

\subsubsection{Тестовые наборы для модуля pipeline.py}
\begin{xltabular}{\textwidth}{|X|X|X|}
	\caption{Тестовые наборы для функции \texttt{process\_images} (pipeline.py) \label{tab:test_pipeline}} \\
	\hline
	\centrow Описание теста &
	\centrow Входные данные &
	\centrow Эталонные выходные данные \\
	\hline
	\endfirsthead
	
	\caption*{Продолжение таблицы \ref{tab:test_pipeline}} \\
	\hline
	\centrow Описание теста &
	\centrow Входные данные &
	\centrow Эталонные выходные данные \\
	\hline
	\endhead
	
	Проверка генерации 3 аугментированных изображений. & Изображение $100 \times 100$ пикселей, volume\_level='1х25' с изменением на 3. & Список из 3 аугментированных изображений размером $100 \times 100$ пикселей. \\ \hline
	Проверка корректности именования файлов. & Изображение $100 \times 100$ пикселей, volume\_level='1х25', сохранение в папку. & Файлы с именами вроде test\_aug\_001.png, test\_aug\_002.png, test\_aug\_003.png. \\ \hline
\end{xltabular}

\subsubsection{Тестовые наборы для модуля main\_window.py}
\begin{xltabular}{\textwidth}{|X|X|X|}
	\caption{Тестовые наборы для класса \texttt{MainWindow} (main\_window.py) \label{tab:test_main_window}} \\
	\hline
	\centrow Описание теста &
	\centrow Входные данные &
	\centrow Эталонные выходные данные \\
	\hline
	\endfirsthead
	
	\caption*{Продолжение таблицы \ref{tab:test_main_window}} \\
	\hline
	\centrow Описание теста &
	\centrow Входные данные &
	\centrow Эталонные выходные данные \\
	\hline
	\endhead
	
	Проверка загрузки одного изображения. & Папка с одним изображением $100 \times 100$ пикселей. & image\_count = 1, отображение предпросмотра. \\ \hline
	Проверка загрузки нескольких изображений. & Папка с тремя изображениями $100 \times 100$ пикселей. & image\_count = 3, отображение списка предпросмотров. \\ \hline
\end{xltabular}

\subsubsection{Тестовые наборы для модуля augmentation\_selector.py}
\begin{xltabular}{\textwidth}{|X|X|X|}
	\caption{Тестовые наборы для класса \texttt{AugmentationSelectorDialog} (augmentation\_selector.py) \label{tab:test_augmentation_selector}} \\
	\hline
	\centrow Описание теста &
	\centrow Входные данные &
	\centrow Эталонные выходные данные \\
	\hline
	\endfirsthead
	
	\caption*{Продолжение таблицы \ref{tab:test_augmentation_selector}} \\
	\hline
	\centrow Описание теста &
	\centrow Входные данные &
	\centrow Эталонные выходные данные \\
	\hline
	\endhead
	
	Проверка выбора двух аугментаций. & Выбор rotate и shift. & Список ['rotate', 'shift']. \\ \hline
	Проверка отсутствия выбора. & Отсутствие отмеченных чекбоксов. & Пустой список []. \\ \hline
\end{xltabular}

\subsection{Системное тестирование разработанной программной системы}

Для проведения системного тестирования были использованы 5 полутоновых изображений 500x500 пикселей в формате jpg. Аугментация проводилась в масштабе - "1к50".

На рисунке~\ref{fig:systest1} представлено главное окно приложения при запуске.
\begin{figure}[H]
	\centering
	\includegraphics[width=0.4\linewidth]{"images/systest1"}
	\caption{Окно приложения при запуске>}
	\label{fig:systest1}
\end{figure}

На рисунке~\ref{fig:systest2} представлено диалоговое окно выбора папки с иходными изображеними.

\begin{figure}[H]
	\centering
	\includegraphics[width=0.8\linewidth]{"images/systest2"}
	\caption{Диалоговое окно выбора выбора папки с иходными изображеними}
	\label{fig:systest2}
\end{figure}

На рисунке~\ref{fig:systest3} представлено диалоговое окно выбора папки, в которой будут генерироваться аугментированные изображения.
\begin{figure}[H]
	\centering
	\includegraphics[width=0.8\linewidth]{"images/systest3"}
	\caption{Диалоговое окно выбора папки, в которой будут генерироваться аугментированные изображения}
	\label{fig:systest3}
\end{figure}

На рисунке~\ref{fig:systest4} представлено диалоговое окно выбора .
\begin{figure}[H]
	\centering
	\includegraphics[width=0.4\linewidth]{"images/systest4"}
	\caption{Результат распознавания нефтяных пятен}
	\label{fig:systest4}
\end{figure}

На рисунке~\ref{fig:systest5} представлено отображение выбора масштаба аугментации.
\begin{figure}[H]
	\centering
	\includegraphics[width=0.4\linewidth]{"images/systest5"}
	\caption{Отображение выбора масштаба аугментации}
	\label{fig:systest5}
\end{figure}

На рисунке~\ref{fig:systest6} представлено отображение выбора методов аугментации.
\begin{figure}[H]
	\centering
	\includegraphics[width=0.4\linewidth]{"images/systest6"}
	\caption{Отображение выбора методов аугментации}
	\label{fig:systest6}
\end{figure}

На рисунке~\ref{fig:systest7} представлено главное окно приложения при запуске.
\begin{figure}[H]
	\centering
	\includegraphics[width=0.8\linewidth]{"images/systest7"}
	\caption{Результат распознавания нефтяных пятен}
	\label{fig:systest7}
\end{figure}
