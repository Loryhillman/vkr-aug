\section{Рабочий проект}
\subsection{Спецификация компонентов и классов программы}

\subsection{Модуль shift.py}

Модуль shift.py содержит функцию для случайного сдвига изображения по осям X и Y. Модуль не содержит классов. Метод модуля - shift image. Он выполняет случайный сдвиг изображения на расстояние, не превышающее заданный процент от его размеров, с использованием аффинного преобразования. Входные данные:

\begin{itemize}
	\item image (тип PIL.Image.Image) – исходное изображение для сдвига;
	\item max shift (тип float, значение по умолчанию = 0.1) – максимальный процент сдвига от размера изображения.
\end{itemize}

Возвращаемые данные: PIL.Image.Image – изображение после сдвига.

\subsection{Модуль brightness.py}

Модуль brightness.py предоставляет функцию для изменения яркости изображения. Модуль не содержит классов. Метод модуля - change brightness. Он изменяет яркость изображения с использованием класса ImageEnhance.Brightness из библиотеки Pillow. Входные данные:

\begin{itemize}
	\item image (тип PIL.Image.Image) – исходное изображение;
	\item factor (тип float, значение по умолчанию = 1.0) – коэффициент яркости (меньше 1 – затемнение, больше 1 – осветление).
\end{itemize}

Возвращаемые данные: PIL.Image.Image – изображение с измененной яркостью.

\subsection{Модуль rotate.py}

Модуль rotate.py содержит функцию для поворота изображения на заданный угол. Модуль не содержит классов. Метод модуля - rotate image. Он выполняет поворот изображения на заданный угол с возможностью изменения размера после поворота. Входные данные:

\begin{itemize}
	\item image (тип PIL.Image.Image) – исходное изображение;
	\item angle (тип float) – угол поворота в градусах;
	\item target size (тип tuple, необязательный) – целевой размер изображения после поворота (ширина, высота).
\end{itemize}

Возвращаемые данные: PIL.Image.Image – повернутое изображение (и измененного размера, если указан target size).

\subsection{Модуль flip.py}

Модуль flip.py предоставляет функцию для отражения изображения по горизонтали или вертикали. Модуль не содержит классов. Метод модуля - flip image. Он отражает изображение по горизонтали или вертикали с использованием метода transpose из библиотеки Pillow. При некорректном значении параметра mode вызывает исключение ValueError. Входные данные:

\begin{itemize}
	\item image (тип PIL.Image.Image) – исходное изображение;
	\item mode (тип str, значение по умолчанию = 'horizontal') – режим отражения (horizontal или vertical).
\end{itemize}

Возвращаемые данные: PIL.Image.Image – отраженное изображение.

\subsection{Модуль noise.py}

Модуль noise.py содержит функцию для добавления шума к изображению в градациях серого. Модуль не содержит классов. Метод модуля - add noise. Он добавляет к изображению шум типа "гауссовский" или "соль и перец". Для обработки используется библиотека NumPy. Требует, чтобы изображение было в режиме L, иначе вызывает исключение ValueError. Входные данные:

\begin{itemize}
	\item image (тип PIL.Image.Image) – исходное изображение в режиме L (градации серого);
	\item noise type (тип str, значение по умолчанию = 'gaussian') – тип шума (gaussian или salt pepper);
	\item **kwargs – дополнительные параметры:
	\begin{itemize}
		\item для gaussian: stddev (тип float, значение по умолчанию = 10) – стандартное отклонение шума;
		\item для salt pepper: amount (тип float, значение по умолчанию = 0.05) – доля пикселей, затронутых шумом; salt vs pepper (тип float, значение по умолчанию = 0.5) – соотношение "соли" и "перца".
	\end{itemize}
\end{itemize}

Возвращаемые данные: PIL.Image.Image – изображение с добавленным шумом.

\subsection{Модуль augmentation config.py}

Модуль augmentation config.py определяет конфигурацию аугментаций. Модуль не содержит классов или методов, представляя собой словарь augmentation config. Описание конфигурации:

\begin{itemize}
	\item available augmentations (тип list) – список доступных аугментаций: noise, rotate, flip, shift, brightness;
	\item noise, rotate, flip, shift, brightness (тип dict) – настройки для каждой аугментации с полями enabled и params (например, angle range, std range);
	\item volume presets (тип dict) – предустановленные объемы генерации: 1х25, 1х50, 1х100.
\end{itemize}

Он обеспечивает централизованное управление параметрами аугментаций и их состоянием.

\subsection{Модуль pipeline.py}

Модуль pipeline.py реализует логику применения аугментаций. Он обеспечивает централизованное управление параметрами аугментаций и их состоянием. Методы модуля представлены в таблице ~\ref{table:pipeline}.

\renewcommand{\arraystretch}{0.8} % уменьшение расстояний до сетки таблицы
\begin{xltabular}{\textwidth}{|>{\hsize=0.9\hsize\raggedright\arraybackslash}X|
		>{\hsize=1.0\hsize\setlength{\baselineskip}{0.7\baselineskip}}X|
		>{\hsize=1.0\hsize}X|
		>{\hsize=1.3\hsize}X|}
	\caption{Методы модуля pipeline.py\label{table:pipeline}}\\
	\hline 
	\centrow \setlength{\baselineskip}{0.7\baselineskip} Название метода & 
	\centrow Параметры метода &
	\centrow Возвращаемое значение & 
	\centrow Назначение метода \\ 
	\hline 
	\endfirsthead
	
	\caption*{Продолжение таблицы \ref{table:pipeline}}\\
	\hline 
	\centrow Название метода & 
	\centrow Параметры метода &
	\centrow Возвращаемое значение & 
	\centrow Назначение метода \\ 
	\hline 
	\endhead
	
	\_\_init\_\_ & Не имеет & Не имеет  & Инициализирует главное окно программы, задает его параметры, заголовок, панель меню с действиями для переключения режимов \\ \hline 
	
	apply\_ augmentation & image (тип PIL.Image. Image) – изображение; augmentation\_ type (тип str) – тип аугментации & PIL.Image. Image – аугментированное изображение & Применяет одну аугментацию в соответствии с конфигурацией.\\
	\hline
	
	process\_ images & image (тип PIL.Image. Image) – изображение; target\_size (тип tuple) – размер; volume\_level (тип str, по умолчанию 'low') – уровень генерации & list – список аугментированных изображений & Генерирует заданное количество аугментированных версий с случайными комбинациями аугментаций.\\
	\hline
	
	process\_ images & image (тип PIL.Image. Image) – изображение; target\_size (тип tuple) – размер; volume\_level (тип str, по умолчанию 'low') – уровень генерации & list – список аугментированных изображений & Генерирует заданное количество аугментированных версий с случайными комбинациями аугментаций.\\
	\hline
	
\end{xltabular}
\renewcommand{\arraystretch}{1.0} % восстановление сетки
\vspace{-\baselineskip}


\subsection{Модуль main\_window.py}

Модуль main\_window предоставляет графический интерфейс для взаимодействия с пользователем, включая загрузку изображений, выбор директории, настройку аугментаций, предпросмотр и сохранение результатов. Константы и методы: отсутствуют.

Класс MainWindow (модуль main\_window.py)

Базовый класс: AugmentationWindow (из модуля ui.window).

Внутренние поля представлены в таблице ~\ref{table:main_window}

\begin{xltabular}{\textwidth}{|X|X|X|}
	\caption{Внутренние поля класса MainWindow \label{table:main_window}} \\
	\hline 
	\centrow Внутреннее поле & 
	\centrow Тип & 
	\centrow Описание \\ 
	\hline 
	\endfirsthead
	
	\caption*{Продолжение таблицы \ref{table:main_window}} \\
	\hline 
	\centrow Внутреннее поле & 
	\centrow Тип & 
	\centrow Описание \\ 
	\hline 
	\endhead
	
	output\_dir & str или None & Путь к директории для сохранения результатов. \\ \hline
	image\_paths & list & Список путей к загруженным изображениям. \\ \hline
	progress\_bar & QProgressBar & Виджет для отображения прогресса обработки. \\ \hline
	scroll\_area & QScrollArea & Область прокрутки для отображения миниатюр. \\ \hline
	preview\_container & QWidget & Контейнер для размещения миниатюр аугментированных изображений. \\ \hline
	preview\_layout & QHBoxLayout & Макет для размещения миниатюр. \\ \hline
	select\_augs\_button & QPushButton & Кнопка для открытия диалога выбора аугментаций. \\ \hline
\end{xltabular}

Методы класса представлены в таблице ~\ref{table:main_window_method}

\renewcommand{\arraystretch}{0.8} % уменьшение расстояний до сетки таблицы
\begin{xltabular}{\textwidth}{|>{\hsize=0.9\hsize\raggedright\arraybackslash}X|
		>{\hsize=1.0\hsize\setlength{\baselineskip}{0.7\baselineskip}}X|
		>{\hsize=1.0\hsize}X|
		>{\hsize=1.3\hsize}X|}
	\caption{Методы модуля pipeline.py\label{table:main_window_method}}\\
	\hline 
	\centrow \setlength{\baselineskip}{0.7\baselineskip} Название метода & 
	\centrow Параметры метода &
	\centrow Возвращаемое значение & 
	\centrow Назначение метода \\ 
	\hline 
	\endfirsthead
	
	\caption*{Продолжение таблицы \ref{table:main_window_method}}\\
	\hline 
	\centrow Название метода & 
	\centrow Параметры метода &
	\centrow Возвращаемое значение & 
	\centrow Назначение метода \\ 
	\hline 
	\endhead
	
	load\_images & Не имеет & Не имеет  & Загружает изображения из выбранной папки и отображает предпросмотр первого. \\ \hline
	
	select\_output \_folder & Не имеет & Не имеет  & Загружает изображения из выбранной папки и отображает предпросмотр первого. \\ \hline

	open\_output\_ directory & Не имеет & Не имеет  & Открывает папку с результатами в файловом менеджере. \\ \hline 

	show\_image \_preview & path (тип str) – путь к изображению & Не имеет  & Отображает предпросмотр исходного изображения в интерфейсе. \\ \hline 

	show\_ augmented\_ preview & pil\_image (тип PIL.Image.Image) – аугментированное изображение & Не имеет & Отображает предпросмотр аугментированного изображения в интерфейсе. \\ \hline
	
	apply\_ selected\_ augmentation & Не имеет & Не имеет & Выполняет аугментацию для всех загруженных изображений и сохраняет результаты. \\ \hline
	
	open\_ augmentation \_selector & Не имеет & Не имеет & Открывает диалог для настройки активных аугментаций. \\ \hline
	
	apply\_ augmentation & image (тип PIL.Image. Image) – изображение; augmentation\_ type (тип str) – тип аугментации & PIL.Image. Image – аугментированное изображение & Применяет одну аугментацию в соответствии с конфигурацией.\\
	\hline
	
	process\_ images & image (тип PIL.Image. Image) – изображение; target\_size (тип tuple) – размер; volume\_level (тип str, по умолчанию 'low') – уровень генерации & list – список аугментированных изображений & Генерирует заданное количество аугментированных версий с случайными комбинациями аугментаций.\\
	\hline
	
	process\_ images & image (тип PIL.Image. Image) – изображение; target\_size (тип tuple) – размер; volume\_level (тип str, по умолчанию 'low') – уровень генерации & list – список аугментированных изображений & Генерирует заданное количество аугментированных версий с случайными комбинациями аугментаций.\\
	\hline
	
\end{xltabular}
\renewcommand{\arraystretch}{1.0} % восстановление сетки
\vspace{-\baselineskip}

\subsection{Модуль augmentation\_selector.py}

Модуль augmentation\_selector.py предоставляет диалоговое окно для выбора активных аугментаций. Константы: отсутствуют. Метод get\_selected\_augmentations возвращает список аугментаций, отмеченных пользователем.

Внутренние поля представлены в таблице ~\ref{table:table:augmentation_selector}

\begin{xltabular}{\textwidth}{|X|X|X|}
	\caption{Внутренние поля класса MainWindow \label{table:augmentation_selector}} \\
	\hline 
	\centrow Внутреннее поле & 
	\centrow Тип & 
	\centrow Описание \\ 
	\hline 
	\endfirsthead
	
	\caption*{Продолжение таблицы \ref{table:main_window}} \\
	\hline 
	\centrow Внутреннее поле & 
	\centrow Тип & 
	\centrow Описание \\ 
	\hline 
	\endhead
	
	selected & set & Множество выбранных пользователем аугментаций. \\ \hline
	checkboxes & dict & Словарь, где ключ – название аугментации, значение – QCheckBox. \\ \hline
\end{xltabular}

\subsection{Модульное тестирование разработанной программной системы}

Модульное тестирование проведено для проверки корректности работы отдельных компонентов программной системы аугментации изображений. Тестирование осуществлялось с использованием встроенной библиотеки unittest языка Python, что позволило автоматизировать проверки функциональности функций и классов. Для каждого теста указаны назначение, код и результат, полученный при его выполнении. Тесты выполнялись на тестовом изображении размером 100x100 пикселей, созданном программно с использованием библиотеки Pillow, а результаты проверялись автоматически с сохранением файлов для визуального анализа.


Тест 1: Проверка функции сдвига изображения (shift.py)
Назначение теста: Убедиться, что функция shift\_image корректно выполняет случайный сдвиг изображения в пределах заданного диапазона без изменения размеров.

\begin{figure}[H]
	\begin{lstlisting}[language=Python]
		import unittest
		from PIL import Image
		import random
		from augmentor import shift_image
		
		class TestShiftImage(unittest.TestCase):
		def setUp(self):
		
		random.seed(42)
		
		self.test_image = Image.new('RGB', (100, 100), color='white')
		self.test_image.save('test_original_shift.png')
		
		def test_shift_image(self):
		
		shifted_image = shift_image(self.test_image, max_shift=0.1)
		shifted_image.save('test_shifted.png')
		
		self.assertEqual(shifted_image.size, self.test_image.size, "Размеры изображения изменились после сдвига.")
		
		
		if __name__ == '__main__':
		unittest.main()
	\end{lstlisting}  
	\caption{Модульный тест функции shift\_image}
	\label{model_test:test1}
\end{figure}

Результат тестирования:
\begin{figure}[H]
	\centering
	\includegraphics[width=0.7\linewidth]{images/resulttest1}
	\caption{Результат тестирования функции shift\_image}
	\label{fig:resulttest1}
\end{figure}

Тест 2: Проверка функции изменения яркости (brightness.py)
Назначение теста: Проверить, что функция change\_brightness корректно увеличивает яркость изображения при коэффициенте больше 1.

\begin{figure}[H]
	\begin{lstlisting}[language=Python]
		import unittest
		from PIL import Image
		import numpy as np
		from augmentor import change_brightness
		
		class TestBrightness(unittest.TestCase):
		def setUp(self):
		self.test_image = Image.new('RGB', (100, 100), color=(128, 128, 128))
		self.test_image.save('test_original_brightness.png')
		
		def test_change_brightness(self):
		
		bright_image = change_brightness(self.test_image, factor=1.5)
		bright_image.save('test_brightened.png')
		
		bright_array = np.array(bright_image)
		original_array = np.array(self.test_image)
		self.assertGreater(bright_array.mean(), original_array.mean(), "Яркость не увеличилась.")
		
		if __name__ == '__main__':
		unittest.main()



	\end{lstlisting}  
	\caption{Модульный тест функции change\_brightness}
	\label{model_test:test2}
\end{figure}

Результат тестирования:
\begin{figure}[H]
	\centering
	\includegraphics[width=0.7\linewidth]{images/resulttest2}
	\caption{Результат тестирования функции change\_brightness}
	\label{fig:resulttest2}
\end{figure}

Тест 3: Проверка функции поворота изображения (rotate.py)
Назначение теста: Убедиться, что функция rotate\_image корректно поворачивает изображение и сохраняет заданный размер при использовании target\_size.

\begin{figure}[H]
	\begin{lstlisting}[language=Python]
		import unittest
		from PIL import Image
		from augmentor import rotate_image
		
		class TestRotateImage(unittest.TestCase):
		def setUp(self):
		self.test_image = Image.new('RGB', (100, 100), color='white')
		self.test_image.save('test_original_rotate.png')
		
		def test_rotate_image(self):
		
		rotated_image = rotate_image(self.test_image, angle=90, target_size=(100, 100))
		rotated_image.save('test_rotated.png')
		self.assertEqual(rotated_image.size, (100, 100), "Размер изображения не соответствует target_size.")
		
		if __name__ == '__main__':
		unittest.main()
	\end{lstlisting}  
	\caption{Модульный тест функции rotate\_image}
	\label{model_test:test3}
\end{figure}

Результат тестирования:
\begin{figure}[H]
	\centering
	\includegraphics[width=0.7\linewidth]{images/resulttest3}
	\caption{Результат тестирования функции rotate\_image}
	\label{fig:resulttest3}
\end{figure}

Тест 4: Проверка функции отражения изображения (flip.py)
Назначение теста: Проверить, что функция flip\_image корректно отражает изображение по горизонтали.

\begin{figure}[H]
	\begin{lstlisting}[language=Python]
		import unittest
		from PIL import Image
		from augmentor import flip_image
		
		class TestFlipImage(unittest.TestCase):
		def setUp(self):
		self.test_image = Image.new('RGB', (100, 100), color='black')
		self.test_image.save('test_original_flip.png')
		
		def test_flip_image(self):
		
		flipped_image = flip_image(self.test_image, mode='horizontal')
		flipped_image.save('test_flipped.png')
		self.assertEqual(flipped_image.size, self.test_image.size, "Размеры изменились после отражения.")
		
		if __name__ == '__main__':
		unittest.main()
	\end{lstlisting}  
	\caption{Модульный тест функции flip\_image}
	\label{model_test:test4}
\end{figure}

Результат тестирования:
\begin{figure}[H]
	\centering
	\includegraphics[width=0.7\linewidth]{images/resulttest4}
	\caption{Результат тестирования функции flip\_image}
	\label{fig:resulttest4}
\end{figure}

Тест 5: Проверка функции добавления шума (noise.py)
Назначение теста: Убедиться, что функция add\_noise добавляет шум к изображению в градациях серого.

\begin{figure}[H]
	\begin{lstlisting}[language=Python]
		import unittest
		from PIL import Image
		import numpy as np
		import random
		from augmentor import add_noise
		
		class TestNoise(unittest.TestCase):
		def setUp(self):
		random.seed(42)
		self.test_image = Image.new('L', (100, 100), color=128)
		self.test_image.save('test_original_noise.png')
		
		def test_add_noise(self):
		
		noisy_image = add_noise(self.test_image, noise_type='gaussian', stddev=10)
		noisy_image.save('test_noisy.png')
		noisy_array = np.array(noisy_image)
		original_array = np.array(self.test_image)
		self.assertGreater(noisy_array.var(), original_array.var(), "Дисперсия не увеличилась после добавления шума.")
		
		if __name__ == '__main__':
		unittest.main()
	\end{lstlisting}  
	\caption{Модульный тест функции add\_noise}
	\label{model_test:test5}
\end{figure}

Результат тестирования:
\begin{figure}[H]
	\centering
	\includegraphics[width=0.7\linewidth]{images/resulttest5}
	\caption{Результат тестирования функции add\_noise}
	\label{fig:resulttest5}
\end{figure}

Тест 6: Проверка функции обработки изображений (pipeline.py)
Назначение теста: Проверить, что функция process\_images генерирует заданное количество аугментированных изображений.

\begin{figure}[H]
	\begin{lstlisting}[language=Python]
		import unittest
		from PIL import Image
		import os
		from config.augmentation_config import augmentation_config
		from core.pipeline import process_images
		
		class TestPipeline(unittest.TestCase):
		def setUp(self):
		self.test_image = Image.new('RGB', (100, 100), color='white')
		self.test_image.save('test_original_pipeline.png')
		
		augmentation_config['volume_presets']['1х25'] = 3  # Уменьшаем для теста
		
		def test_process_images(self):
		
		results = process_images(self.test_image, target_size=(100, 100), volume_level='1х25')
		for i, img in enumerate(results):
		img.save(f'test_pipeline_aug_{i}.png')
		self.assertEqual(len(results), 3, "Количество сгенерированных изображений не равно 3.")
		
		if __name__ == '__main__':
		unittest.main()
	\end{lstlisting}  
	\caption{Модульный тест функции process\_images}
	\label{model_test:test6}
\end{figure}

Результат тестирования:
\begin{figure}[H]
	\centering
	\includegraphics[width=0.7\linewidth]{images/resulttest6}
	\caption{Результат тестирования функции process\_images}
	\label{fig:resulttest6}
\end{figure}

Тест 7: Проверка загрузки изображений в интерфейсе (main\_window.py)
Назначение теста: Убедиться, что класс MainWindow корректно загружает изображения из папки.

\begin{figure}[H]
	\begin{lstlisting}[language=Python]
		from PIL import Image
		import os
		from ui.main_window import MainWindow
		from PySide6.QtWidgets import QApplication
		
		app = QApplication([])
		
		os.makedirs('test_folder', exist_ok=True)
		test_image = Image.new('RGB', (100, 100), color='white')
		test_image.save('test_folder/test_image.png')
		
		main_window = MainWindow()
		main_window.image_paths = ['test_folder/test_image.png']
		main_window.load_images()
		
		if main_window.image_count == 1:
		print("Тест пройден: одно изображение успешно загружено.")
		else:
		print("Тест провален: изображение не загружено.")
	\end{lstlisting}  
	\caption{Модульный тест класса MainWindow}
	\label{model_test:test7}
\end{figure}

Результат тестирования:
\begin{figure}[H]
	\centering
	\includegraphics[width=0.7\linewidth]{images/resulttest7}
	\caption{Результат тестирования класса MainWindow}
	\label{fig:resulttest7}
\end{figure}

Тест 8: Проверка выбора аугментаций в диалоге (augmentation\_selector.py)
Назначение теста: Проверить, что класс AugmentationSelectorDialog корректно сохраняет выбранные пользователем аугментации.

\begin{figure}[H]
	\begin{lstlisting}[language=Python]
		import unittest
		from PySide6.QtWidgets import QApplication
		from ui.augmentation_selector import AugmentationSelectorDialog
		from config.augmentation_config import augmentation_config
		import sys
		
		class TestAugmentationSelector(unittest.TestCase):
		@classmethod
		def setUpClass(cls):
		
		cls.app = QApplication(sys.argv)
		
		def setUp(self):
		
		self.available = augmentation_config['available_augmentations']
		self.selected = []
		self.dialog = AugmentationSelectorDialog(self.available, self.selected)
		
		def test_get_selected_augmentations(self):
		
		self.dialog.checkboxes['rotate'].setChecked(True)
		self.dialog.checkboxes['shift'].setChecked(True)
		selected_augs = self.dialog.get_selected_augmentations()
		
		self.assertIn('rotate', selected_augs, "Аугментация 'rotate' не выбрана.")
		self.assertIn('shift', selected_augs, "Аугментация 'shift' не выбрана.")
		self.assertEqual(len(selected_augs), 2, "Количество выбранных аугментаций не соответствует ожидаемому (2).")
		
		@classmethod
		def tearDownClass(cls):
		
		cls.app.quit()
		
		if __name__ == '__main__':
		unittest.main()
	\end{lstlisting}  
	\caption{Модульный тест класса AugmentationSelectorDialog}
	\label{model_test:test8}
\end{figure}

Результат тестирования:
\begin{figure}[H]
	\centering
	\includegraphics[width=0.7\linewidth]{images/resulttest8}
	\caption{Результат тестирования класса AugmentationSelectorDialog}
	\label{fig:resulttest8}
\end{figure}

\subsection{Системное тестирование разработанной программной системы}

Для проведения системного тестирования были использованы 5 полутоновых изображений 500x500 пикселей в формате jpg. Аугментация проводилась в масштабе - "1к50".

На рисунке~\ref{fig:systest1} представлено главное окно приложения при запуске.
\begin{figure}[H]
	\centering
	\includegraphics[width=0.3\linewidth]{"images/systest1"}
	\caption{Окно приложения при запуске>}
	\label{fig:systest1}
\end{figure}

На рисунке~\ref{fig:systest2} представлено диалоговое окно выбора папки с иходными изображеними.

\begin{figure}[H]
	\centering
	\includegraphics[width=0.5\linewidth]{"images/systest2"}
	\caption{Диалоговое окно выбора выбора папки с иходными изображеними}
	\label{fig:systest2}
\end{figure}

На рисунке~\ref{fig:systest3} представлено диалоговое окно выбора папки, в которой будут генерироваться аугментированные изображения.
\begin{figure}[H]
	\centering
	\includegraphics[width=0.5\linewidth]{"images/systest3"}
	\caption{Диалоговое окно выбора папки, в которой будут генерироваться аугментированные изображения}
	\label{fig:systest3}
\end{figure}

На рисунке~\ref{fig:systest4} представлено диалоговое окно выбора .
\begin{figure}[H]
	\centering
	\includegraphics[width=0.3\linewidth]{"images/systest4"}
	\caption{Результат распознавания нефтяных пятен}
	\label{fig:systest4}
\end{figure}

На рисунке~\ref{fig:systest5} представлено отображение выбора масштаба аугментации.
\begin{figure}[H]
	\centering
	\includegraphics[width=0.3\linewidth]{"images/systest5"}
	\caption{Отображение выбора масштаба аугментации}
	\label{fig:systest5}
\end{figure}

На рисунке~\ref{fig:systest6} представлено отображение выбора методов аугментации.
\begin{figure}[H]
	\centering
	\includegraphics[width=0.3\linewidth]{"images/systest6"}
	\caption{Отображение выбора методов аугментации}
	\label{fig:systest6}
\end{figure}

На рисунке~\ref{fig:systest7} представлено главное окно приложения при запуске.
\begin{figure}[H]
	\centering
	\includegraphics[width=0.5\linewidth]{"images/systest7"}
	\caption{Результат распознавания нефтяных пятен}
	\label{fig:systest7}
\end{figure}
