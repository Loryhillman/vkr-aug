\section{Технический проект}
\subsection{Общая характеристика организации решения задачи}

Необходимо спроектировать и разработать сайт, который должен способствовать продвижению компании на рынке.
Основной задачей системы является автоматизированная обработка изображений, загруженных пользователем из выбранной директории, с последующим применением набора предопределённых преобразований (аугментаций) и сохранением полученных результатов в отдельную директорию. Пользователь должен иметь возможность выбрать уровень интенсивности аугментации (низкий, средний, высокий), задать папку для сохранения и просматривать результаты преобразования в виде предварительного просмотра.
Обработка изображений осуществляется на стороне клиента, без необходимости подключения к сети Интернет. Это позволяет обеспечить максимальную независимость и приватность данных. Программа ориентирована на стабильную работу в среде Windows/Linux с минимальными системными требованиями и без фоновых сетевых процессов.

\subsection{Обоснование выбора технологий проектирования}

Используемые при разработке технологии отвечают современным стандартам и обеспечивают надёжность, удобство сопровождения и масштабируемость проекта. Для построения графического интерфейса выбрана кросс-платформенная библиотека PySide6, а для работы с изображениями — проверенная временем библиотека Pillow.

\subsubsection{Язык программирования Python}

Язык программирования Python был выбран как основной инструмент для реализации системы в силу следующих причин:

\begin{itemize}
	\item Простота синтаксиса, позволяющая ускорить этапы разработки и отладки.
	\item Широкая экосистема библиотек, включая Pillow для работы с изображениями и PySide6 для построения графического интерфейса.
	\item Хорошая читаемость и расширяемость кода, что важно для последующего сопровождения проекта или его масштабирования.
	\item Поддержка мультиплатформенности: программа может быть развёрнута как на Windows, так и на Linux, без необходимости значительных изменений в коде.
	\item Активное сообщество и постоянное развитие языка и его инструментов.
\end{itemize}

Python используется как для создания пользовательского интерфейса, так и для реализации логики обработки изображений, что позволяет обеспечить целостность архитектуры и снизить сложность сопровождения программной системы.

\subsubsection{Исследование и выбор методов аугментации изображений}

Аугментация изображений (image augmentation) представляет собой процесс искусственного увеличения объема обучающей выборки путём внесения различных преобразований в исходные изображения. Актуальность применения аугментации особенно высока при работе с ограниченным числом изображений, в том числе для задач классификации, детекции и сегментации.

Цели внедрения аугментации в рамках настоящего программного решения:

\begin{itemize}
	\item увеличение разнообразия обучающей выборки без реального увеличения числа изображений;
	\item снижение переобучения алгоритмов машинного обучения при последующем применении данных;
	\item имитирование реальных условий съёмки, включая различия в освещении, положении камеры, шумовых помехах и т.д.;
	\item тестирование устойчивости алгоритмов распознавания и аналитики к искажениям;
	\item поддержка регуляризации и улучшение обобщающей способности обучаемых моделей.
\end{itemize}

%РИСУНОК
%Изображение сетка 2×3: оригинал и пять результатов (rotate, flip, noise, brightness, shift). 

Процесс отбора подходящих методов аугментации был организован в несколько этапов:

\begin{enumerate}
	\item Анализ имеющихся методов аугментации, реализуемых через библиотеки PIL, OpenCV и imgaug. Рассматривались следующие трансформации: геометрические (повороты, масштабирование, сдвиги, отражения), композиционные (обрезка, наложение).
	\item Формирование предварительного пула методов: поворот, шум (гауссовский, «соль и перец»), отражение, сдвиг, изменение яркости, обрезка, изменение контрастности.
	\item Применение методов к тестовой выборке (около 100 изображений), анализ визуальных изменений, а также оценка влияния на читаемость структуры изображения.
	\item Формализация критериев отбора: сохранение семантики изображения (разборчивость объектов), воспроизводимость и параметризация, реальная применимость к задачам распознавания и визуального анализа.
	\item Исключение методов, создающих риск искажения семантики (например, scale и агрессивный crop) или плохо применимых в условиях ограниченных вычислительных ресурсов.
\end{enumerate}

%таблица с видами аугментации

Таким образом, отбор прошли пять ключевых методов, которые соответствуют критериям качества, скорости и реалистичности: rotate, noise, flip, shift, brightness.

\subsubsection{Описание и исследование выбранных методов аугментации}

Для достижения наилучшего эффекта преобразования изображений были выбраны пять методов. В данном разделе проводится их подробный анализ, в том числе: принцип работы, исследованные диапазоны параметров, визуальные результаты и рекомендации по применению.

\begin{enumerate}
	\item Поворот изображения

Поворот изображения на определённый угол позволяет имитировать изменение положения камеры или объекта. Преобразование сохраняет геометрию объектов и является одним из наименее деструктивных способов аугментации.

Диапазон параметров:
\begin{itemize}
	\item углы поворота: от -25 до +25 градусов;
	\item шаг исследуемых значений: 5 градусов.
\end{itemize}

Углы выше 30° приводят к искажению восприятия, особенно при наличии симметричных объектов. Было установлено, что диапазон ±25° обеспечивает визуальное разнообразие без потери информативности.

%ИЗОБРАЖЕНИЕ И ВАРИАНТЫ ПОВОРОТА -15 0 И 15 ГРАДУСОВ

	\item Добавление шума
	
Добавление шумов позволяет имитировать условия плохой съёмки, передаёт реалистичность восприятия, особенно для задач, где необходимо устойчивое поведение к зашумлённым данным. Рассматривались два типа: гауссовский шум и соль и перец.

Диапазон параметров:
\begin{itemize}
	\item Среднеквадратичное отклонение (гауссов шум): 2-25
	\item Соль и перец(процент испорченных пикселей): 0.01 - 0.10 (1-10\%)
\end{itemize}

При превышении указанных значений изображение становится трудноузнаваемым. Оптимальные значения были выбраны методом экспертной оценки и визуального тестирования.

%ИЗБОБРАЖЕНИЕ
% Примеры зашумления изображений с разной интенсивностью

	\item Отражение
	
Отражение изображения по горизонтали или вертикали. Простой способ создания новых примеров без искажения формы.

Типы:

\begin{itemize}
	\item Горизонтальное отражение - оптимальное
	\item Вертикальное отражение (не рекомендуется, так как может изменить семантику изображения (например, в случае с текстом или направленными объектами)
\end{itemize}

%ИЗОБРАЖЕНИЕ оригинал и отраженное изображение

	\item Сдвиг

Сдвиг изображения по горизонтали и вертикали, позволяющий сместить объект в пределах кадра, не нарушая общей структуры.

Диапазон параметров:

Смещение по X и Y: от -15\% до +15\% от размера изображения

Смещения более чем на 20\% влекут за собой обрезку ключевых областей. Оптимальные значения позволяют сохранить композицию при создании новой геометрии расположения объектов.

%ИЗОБРАЖЕНИЕ
%смещение по горзонтали +-10%


	\item Изменение яркости

Изменение яркости изображения имитирует освещение в различных условиях (день, ночь, контровой свет и т.п.). Это особенно важно для универсализации восприятия моделей.

Диапазон параметров: коэффициент изменения яркости: 0.6 – 1.4
(где 1.0 — исходное значение)

Значения ниже 0.5 делают изображение почти чёрным, выше 1.5 — засвеченным. Оптимальный диапазон обеспечивает баланс между реализмом и информативностью.

% ИЗОБРАЖЕНИЕ
% зимененение яркости 0.6, 1.0, 1.4

\end{enumerate}

%таблица выбранных оптимальных значений

\subsubsection{Конфигурация уровней аугментации и автоматизация параметров}

Для повышения гибкости и упрощения взаимодействия с системой аугментации в пользовательском интерфейсе были реализованы три уровня интенсивности преобразований: low, medium и high. Каждый уровень соответствует предопределённому набору параметров, автоматически применяемых к каждому изображению.

\textbf{Обоснование уровней}

Уровни были выделены на основе эмпирического анализа, где оценивалась степень трансформации и её влияние на восприятие изображения. % На таблице ниже представлены сводные настройки для каждого уровня
\begin{itemize}
	\item low - отражение, яркость
	\item medium - отражение, яркость, поворот, шум
	\item high - отражение, яркость, поворот, шум, сдвиг
\end{itemize}
	
%ИЗОБРАЖЕНИЕ
%Сравнение изображений при применении уровней low, medium, high.

Система построена по принципу передачи параметров через уровень аугментации. Пользователь выбирает нужный уровень в интерфейсе, и далее программа сопоставляет уровень с конфигурацией (словарь параметров), применяет соответствующие методы аугментации, обеспечивает единообразное поведение для всех изображений партии. Выделение уровней аугментации: упрощает пользовательский опыт, обеспечивает стабильность качества результатов и позволяет масштабировать систему без изменений в логике преобразований.

Для обоснования применимости выбранных методов аугментации и их параметров была проведена оценка качества сгенерированных изображений по следующим критериям:
\begin{itemize}
	\item визуальная читаемость изображения (отсутствие чрезмерных искажений);
	\item сохранение признаков исходного объекта (не теряются ключевые контуры, формы, текстуры);
	\item разнообразие выборки (каждая аугментация должна вносить полезную вариативность, а не избыточный шум);
	\item устойчивость к переобучению на основе опыта использования в системах распознавания (принцип увеличения генерализации).
\end{itemize}

На основании анализа параметров можно выделить следующие оптимальные значения:

\begin{itemize}
	\item поворот: +-15 градусов (максимум 25 градусов)
	\item гауссов шум: 10-15
	\item соль/Перец: 3-5\%
	\item яркость: от 0.8 до 1.2
	\item до 10\% по X и Y
	\item Горизонтальное
\end{itemize}
