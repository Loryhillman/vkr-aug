\section*{ЗАКЛЮЧЕНИЕ}
\addcontentsline{toc}{section}{ЗАКЛЮЧЕНИЕ}

Развитие технологий компьютерного зрения и глубокого обучения открывает новые возможности для создания интеллектуальных систем, способных решать сложные задачи в различных областях, таких как медицина, автономный транспорт и спутниковый мониторинг. Одним из ключевых факторов, обеспечивающих высокую точность и надежность таких систем, является наличие больших и разнообразных наборов данных для обучения.

В рамках данной работы была разработана программа для аугментации изображений с графическим интерфейсом, предназначенная для повышения эффективности обучения нейронных сетей.

Основные результаты работы:

\begin{enumerate}
	\item Проведен анализ предметной области. Изучены методы аугментации изображений и их применение в задачах компьютерного зрения для повышения качества обучения нейронных сетей.
	\item Разработано техническое задание на программу, определены требования к функционалу, интерфейсу и настройкам аугментации.
	\item Осуществлено проектирование программы. Разработана модульная архитектура с поддержкой пакетной обработки и графическим интерфейсом пользователя.
	\item Реализована программа на Python с использованием библиотек Pillow и PySide6, проведено модульное тестирование с применением библиотеки unittest.
\end{enumerate}

Все требования, объявленные в техническом задании, были полностью реализованы, все задачи, поставленные в начале разработки проекта, были также решены.

Готовый проект представлен в виде настольного приложения с графическим интерфейсом, обеспечивающим удобство использования и высокую степень автоматизации процесса аугментации изображений.
