\abstract{РЕФЕРАТ}

Объем работы равен \formbytotal{lastpage}{страниц}{е}{ам}{ам}. Работа содержит \formbytotal{figurecnt}{иллюстраци}{ю}{и}{й}, \formbytotal{tablecnt}{таблиц}{у}{ы}{}, \arabic{bibcount} библиографических источников и \formbytotal{числоПлакатов}{лист}{}{а}{ов} графического материала. Количество приложений – 2. Графический материал представлен в приложении А. Фрагменты исходного кода представлены в приложении Б.

Перечень ключевых слов: аугментация изображений, компьютерное зрение, нейронные сети, Python, OpenCV, Albumentations, геометрические преобразования, цветовые искажения, шум, графический интерфейс, пакетная обработка, машинное обучение, данные, интеллектуальные системы.

Объектом разработки являются автоматизированные системы обработки и распознавания изображений.

Предметом исследования является разработка программы для размножения и аугментации изображений с графическим интерфейсом для компании ООО «Предприятие ВТИ-Сервис». 

Целью выпускной квалификационной работы является повышение эффективности обучения нейронных сетей за счет автоматизированной генерации разнообразных обучающих данных.

В процессе создания программы были выделены ключевые типы аугментаций (геометрические, цветовые, шумовые) через создание соответствующих модулей обработки изображений. Реализованы классы и методы для управления параметрами преобразований (угол поворота, уровень шума, коэффициент контраста), а также разработан механизм пакетной обработки, обеспечивающий корректную работу с малыми наборами данных (менее 10 изображений). Создан интерфейс для настройки уровней аугментации (1×25, 1×50, 1×100) и предпросмотра результатов. Дополнительно реализован функционал экспорта обработанных изображений в стандартных форматах (PNG, JPEG).

Для разработки использованы: язык программирования Python, библиотеки OpenCV и PyQt.

Разработанный сайт был успешно внедрен в компании.

\selectlanguage{english}
\abstract{ABSTRACT}
  
The volume of work is \formbytotal{lastpage}{page}{}{s}{s}. The work contains \formbytotal{figurecnt}{illustration}{}{s}{s}, \formbytotal{tablecnt}{table}{}{s}{s}, \arabic{bibcount} bibliographic sources and \formbytotal{числоПлакатов}{sheet}{}{s}{s} of graphic material. The number of applications is 2. The graphic material is presented in annex A. The layout of the site, including the connection of components, is presented in annex B.

List of keywords: image augmentation, computer vision, neural networks, Python, OpenCV, Albumentations, geometric transformations, color distortions, noise, GUI, batch processing, machine learning, data, intelligent systems.

The object of the research is an image augmentation software with a graphical interface for ООО «Предприятие ВТИ-Сервис».

The object of the development is to improve the training efficiency of neural networks through automated generation of diverse training data.


In the development process, key augmentation types (geометрические, цветовые, шумовые) were implemented through dedicated image processing modules. The system includes classes and methods for transformation parameter control (rotation angle, noise level, contrast ratio) along with a batch processing mechanism that ensures proper operation with small datasets (under 10 images). The interface allows configuring augmentation levels (1×25, 1×50, 1×100) and previewing results. Additionally, the export functionality for processed images in standard formats (PNG, JPEG) has been implemented.

The developed website was successfully implemented in the company.
\selectlanguage{russian}
