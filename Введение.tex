\section*{ВВЕДЕНИЕ}
\addcontentsline{toc}{section}{ВВЕДЕНИЕ}

Методы компьютерного зрения и глубокого обучения стали неотъемлемой частью современных интеллектуальных систем. Однако их эффективность напрямую зависит от объема и качества обучающих данных. Аугментация изображений — ключевой инструмент для решения проблемы нехватки данных, позволяя искусственно расширять выборки за счет преобразований (геометрических, цветовых, шумовых). Интенсивное развитие нейронных сетей, особенно в областях медицины, автономного транспорта и спутникового мониторинга, требует универсальных программных решений для генерации разнообразных данных.

Аугментация предполагает автоматизированное создание вариаций изображений без изменения их семантического содержания, в отличие от ручного сбора дополнительных данных. Современные библиотеки (например, Albumentations, OpenCV) предоставляют технические возможности для таких преобразований, но их использование часто требует навыков программирования.

Современные ИТ-тренды показывают, что автоматизация подготовки данных — обязательный этап в pipeline машинного обучения. Программа для аугментации становится «лицом» процесса обработки изображений, так как именно она превращает ограниченные исходные данные в пригодные для обучения модели. Это напрямую влияет на точность и надежность интеллектуальных систем.

\emph{Цель настоящей работы} — разработка программы аугментации изображений с графическим интерфейсом для повышения эффективности обучения нейронных сетей. Для достижения цели необходимо решить \emph{следующие задачи}:
\begin{itemize}
	\item провести анализ методов аугментации и их применимости в задачах компьютерного зрения;
	\item разработать техническое задание на ПО, учитывающее требования пользователей;
	\item спроектировать архитектуру программы с поддержкой пакетной обработки и настройки параметров;
	\item реализовать программу на Python с использованием библиотек OpenCV и PyQt;
\end{itemize}

\emph{Структура и объем работы.} Отчёт состоит из введения, трёх разделов основной части, заключения, списка использованных источников и приложений. Общий объём работы — \formbytotal{lastpage}{страниц}{у}{ы}{}.

\emph{Во введении} обоснована актуальность темы, сформулированы цель и задачи работы, описана её структура.

\emph{В первом разделе} проведён анализ предметной области: рассмотрены принципы работы нейронных сетей, методы аугментации изображений и их применение в прикладных задачах (медицина, агротехнологии, безопасность).

\emph{Во втором разделе} представлено техническое задание на разработку программы, включая требования к интерфейсу, функционалу (загрузка изображений, выбор уровня аугментации, предпросмотр) и данным.

\emph{В третьем разделе} описаны проектные решения: выбор технологий (Python, OpenCV), конфигурация уровней аугментации (1×25, 1×50, 1×100) и реализация модулей.

В \emph{заключении} подведены итоги работы, оценена практическая значимость программы и направления её дальнейшего развития.

В \emph{приложении А} приведены примеры аугментированных изображений.
В \emph{приложении Б} представлены фрагменты исходного кода программы. 
