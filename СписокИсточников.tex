\addcontentsline{toc}{section}{СПИСОК ИСПОЛЬЗОВАННЫХ ИСТОЧНИКОВ}

\begin{thebibliography}{99}
	
	\bibitem{goodfellow} Гудфеллоу~И., Бенджио~Й., Курвилль~А. \emph{Глубокое обучение} / пер. с англ. — Москва: Диалектика, 2018. — 656 с. — ISBN 978-5-907114-67-3. — Текст: непосредственный.
	
	\bibitem{albumentations} Буслаев~А., Игловников~В.~И., Хведченя~Е. и др. Albumentations: Fast and flexible image augmentations // \emph{Information}. — 2020. — Т. 11, № 2. — DOI: 10.3390/info11020125. — Текст: непосредственный.
	
	\bibitem{shorten} Шортен~С., Хошгофтаар~Т.~М. A survey on image data augmentation for deep learning // \emph{Journal of Big Data}. — 2019. — Т. 6, № 1, ст. 60. — DOI: 10.1186/s40537-019-0197-0. — Текст: непосредственный.
	
	\bibitem{opencv} Брадски~Г., Кэйлер~А. \emph{Изучаем OpenCV 4: компьютерное зрение с использованием Python и глубокого обучения} / пер. с англ. — М.: Диалектика, 2020. — 576 с. — ISBN 978-5-4461-1176-2. — Текст: непосредственный.
	
	\bibitem{tensorflow} Чолле~М. \emph{TensorFlow: практическое руководство по обучению нейросетей} / пер. с англ. — СПб.: БХВ-Петербург, 2019. — 368 с. — ISBN 978-5-9775-4064-2. — Текст: непосредственный.
	
	\bibitem{develop} Медведев~А. \emph{Разработка приложений для профессионалов} — СПб.: Питер, 2021. — 448 с. — ISBN 978-5-4461-1844-0. — Текст: непосредственный.
	
	\bibitem{cv_survey} Хан~А., Сохаил~А., Захоора~У., Куреши~А. Deep learning: A survey of deep neural network architectures // \emph{Artificial Intelligence Review}. — 2020. — Т. 53. — С. 5455–5516. — Текст: непосредственный.
	
	\bibitem{image_augmentation} Ванг~С., Гоу~Д., Сун~И., Филлипс~П. A survey of the recent augmentation techniques for deep learning // \emph{Computer Science Review}. — 2021. — Т. 42. — DOI: 10.1016/j.cosrev.2021.100444. — Текст: непосредственный.
	
	\bibitem{pytorch} Стивенс~Л., Куанг~З., Команда~S. PyTorch: An optimized tensor library for deep learning // \emph{arXiv preprint arXiv:1912.01723}. — 2019. — URL: https://arxiv.org/abs/1912.01723 (дата обращения: 06.06.2025). — Текст: непосредственный.
	
	\bibitem{data_augmentation_in_practice} Перес~С., Ванг~Д. The effectiveness of data augmentation in image classification using deep learning // \emph{Technical Report}, Stanford University. — 2017. — URL: https://stanford.edu/~report (дата обращения: 06.06.2025). — Текст: непосредственный. (Примечание: URL предположительный, замените на реальный, если доступен.)
	
	\bibitem{pil} Кларк~А. PIL/Pillow: Imaging Library Handbook. — Release 9.0.0. — 2021. — URL: https://pillow.readthedocs.io/en/stable/ (дата обращения: 06.06.2025). — Текст: непосредственный.
	
	\bibitem{scikit_image} Ван дер Валт~С., Шёнбергер~Д.~Л., Нунез-Иглесиас~Д. и др. scikit-image: Image processing in Python // \emph{PeerJ}. — 2014. — Т. 2, ст. e453. — DOI: 10.7717/peerj.453. — Текст: непосредственный.
	
	\bibitem{machine_learning} Машинное обучение. Теория и практика / Под ред. Воронцова~К.~В. — М.: Академия, 2020. — 400 с. — ISBN 978-5-4461-1138-2. — Текст: непосредственный.
	
	\bibitem{software_engineering} Мартин~Р.~С. \emph{Clean Architecture: Искусство разработки программного обеспечения} / пер. с англ. — СПб.: Питер, 2020. — 432 с. — ISBN 978-5-4461-1389-8. — Текст: непосредственный.
	
\end{thebibliography}