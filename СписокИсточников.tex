\begin{thebibliography}{99}
	
	\bibitem{goodfellow} Гудфеллоу И., Бенджио Й., Курвилль А. \emph{Глубокое обучение}. — Москва~: Диалектика, 2018. — 656 с. — ISBN 978-5-907114-67-3.
	
	\bibitem{albumentations} Buslaev A., Iglovikov V. I., Khvedchenya E. et al. Albumentations: Fast and flexible image augmentations // \emph{Information}. — 2020. — Vol. 11(2). — DOI: 10.3390/info11020125.
	
	\bibitem{shorten} Shorten C., Khoshgoftaar T. M. A survey on image data augmentation for deep learning // \emph{Journal of Big Data}. — 2019. — Vol. 6(1):60. — DOI: 10.1186/s40537-019-0197-0.
	
	\bibitem{opencv} Брадски Г., Кэйлер А. \emph{Изучаем OpenCV 4: компьютерное зрение с использованием Python и глубокого обучения}. — М.~: Диалектика, 2020. — 576 с. — ISBN 978-5-4461-1176-2.
	
	\bibitem{tensorflow} Чолле М. \emph{TensorFlow: практическое руководство по обучению нейросетей}. — СПб~: БХВ-Петербург, 2019. — 368 с. — ISBN 978-5-9775-4064-2.
	
	\bibitem{develop} Медведев А. \emph{Разработка приложений для профессионалов}. — СПб~: Питер, 2021. — 448 с. — ISBN 978-5-4461-1844-0.
	
	\bibitem{cv_survey} Khan A., Sohail A., Zahoora U., Qureshi A. Deep learning: A survey of deep neural network architectures // \emph{Artificial Intelligence Review}. — 2020. — Vol. 53. — pp. 5455–5516.
	
	\bibitem{image_augmentation} Wang C., Gou J., Sun Y., Phillips P. A survey of the recent augmentation techniques for deep learning // \emph{Computer Science Review}. — 2021. — Vol. 42. — DOI: 10.1016/j.cosrev.2021.100444.
	
	\bibitem{pytorch} Stevens L., Kuang Z., Team S. PyTorch: An optimized tensor library for deep learning // \emph{arXiv preprint arXiv:1912.01723}, 2019.
	
	\bibitem{data_augmentation_in_practice} Perez S., Wang J. The effectiveness of data augmentation in image classification using deep learning // \emph{Technical Report}, Stanford University, 2017.
	
	\bibitem{gan_augmentation} Zhang H., Odena A., Erickson N. Data Augmentation via Unsupervised Pretraining for Image Classification // \emph{NeurIPS Workshop}, 2017.
	
	\bibitem{pil} Clark A. PIL/Pillow: Imaging Library Handbook. — Release 9.0.0. — 2021.
	
	\bibitem{scikit_image} van der Walt S., Schönberger J. L., Nunez-Iglesias J. et al. scikit-image: Image processing in Python // \emph{PeerJ}, 2014. — Vol. 2:e453. — DOI: 10.7717/peerj.453.
	
	\bibitem{machine_learning} Машинное обучение. Теория и практика / Под ред. Воронцова К. В. — М.~: Академия, 2020. — 400 с. — ISBN 978-5-4461-1138-2.
	
	\bibitem{software_engineering} Martin R. C. \emph{Clean Architecture: Искусство разработки программного обеспечения}. — СПб~: Питер, 2020. — 432 с. — ISBN 978-5-4461-1389-8.
	
\end{thebibliography}
