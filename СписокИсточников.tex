\begin{thebibliography}{99}
	
	\bibitem{goodfellow} Гудфеллоу И., Бенджио Й., Курвилль А. Глубокое обучение. — Москва~: Диалектика, 2018. — 656 с. — ISBN 978-5-907114-67-3.
	
	\bibitem{albumentations} Buslaev A., Iglovikov V. I., Khvedchenya E. et al. Albumentations: Fast and flexible image augmentations // *Information*. — 2020. — Vol. 11(2). — DOI: 10.3390/info11020125.
	
	\bibitem{shorten} Shorten C., Khoshgoftaar T. M. A survey on image data augmentation for deep learning // *Journal of Big Data*. — 2019. — Vol. 6(1):60. — DOI: 10.1186/s40537-019-0197-0.
	
	\bibitem{opencv} Брадски Г., Кэйлер А. Изучаем OpenCV 4: компьютерное зрение с использованием Python и глубокого обучения. — М.~: Диалектика, 2020. — 576 с. — ISBN 978-5-4461-1176-2.
	
	\bibitem{tensorflow} Чолле М. TensorFlow: практическое руководство по обучению нейросетей. — СПб~: БХВ-Петербург, 2019. — 368 с. — ISBN 978-5-9775-4064-2.
	
	\bibitem{develop} Медведев А. Разработка приложений для профессионалов. — СПб~: Питер, 2021. — 448 с. — ISBN 978-5-4461-1844-0.
	
	\bibitem{android_battery} Google Developers. Optimizing Battery Life in Android Applications. — [Электронный ресурс]. — URL: \url{https://developer.android.com/topic/performance/power} (дата обращения: 10.05.2025).
	
	\bibitem{background_android} Android Developers. Background Work Overview. — [Электронный ресурс]. — URL: \url{https://developer.android.com/guide/background} (дата обращения: 10.05.2025).
	
	\bibitem{cv_survey} Khan A., Sohail A., Zahoora U., Qureshi A. Deep learning: A survey of deep neural network architectures // *Artificial Intelligence Review*. — 2020. — Vol. 53. — pp. 5455–5516.
	
\end{thebibliography}
